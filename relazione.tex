\documentclass[12pt,a4paper]{article}

\usepackage[main=italian,provide=*]{babel}
\usepackage{graphicx}
\usepackage{amsmath}
\usepackage{hyperref}
\usepackage{geometry}
\usepackage{float}
\usepackage{colortbl}
\usepackage{xcolor}
\geometry{margin=2.5cm}

\setlength{\parindent}{0pt}

\title{Progettazione Database - Noleggio Bici}
\author{Claudio Raimondi}
\date{\today}

\begin{document}

\maketitle

\tableofcontents
\newpage

\section{Richiesta}

Un negozio di noleggio biciclette richiede una base di dati per la gestione operativa della propria flotta, dei clienti e delle transazioni di noleggio effettuate.

\medskip

Ogni bicicletta è identificata univocamente da un codice telaio ed è caratterizzata da specifiche proprietà tecniche: la taglia (bambino, media o grande), il genere (uomo o donna) e la tipologia (mountain bike, city bike o elettrica). Per ciascun mezzo, il sistema deve monitorare costantemente lo stato di disponibilità (libera, noleggiata o in manutenzione). Ogni bicicletta è inoltre equipaggiata con un dispositivo GPS (identificato dal suo numero di serie) e dispone di una nota danni preesistenti per evitare contestazioni alla riconsegna.

\medskip

Le biciclette elettriche richiedono una gestione specifica: per esse è necessario registrare lo stato di carica della batteria sia al momento della consegna, sia al momento della restituzione, per far eventualmente pagare la differenza di carica al cliente.

\medskip

Il negozio interagisce con i clienti, i quali sono identificati dal proprio codice fiscale e dei quali si conservano le generalità come nome, cognome e un recapito telefonico. Il cliente può sottoscrivere uno o più noleggi nel tempo. Ogni noleggio associa formalmente una bicicletta a un cliente per un determinato periodo di utilizzo.

\medskip

In fase di stipula del contratto, viene applicata una tariffa che dipende dalla durata del noleggio e dal tipo di bici. La durata del noleggio segue delle formule standard: giornaliera, settimanale o mensile. Per ogni operazione conclusa, il sistema deve tenere traccia dei dettagli del pagamento, memorizzando il codice ricevuta emesso e la modalità di pagamento (contanti, carta o bitcoin).

\section{Progettazione Concettuale}

\subsection{Analisi Richiesta}

Di seguito si analizza la richiesta evidenziando i termini chiave.

\medskip

Ogni \underline{bicicletta} è identificata univocamente da un \underline{codice telaio} ed è caratterizzata da specifiche proprietà tecniche: la \underline{taglia} (bambino, media o grande), il \underline{genere} (uomo o donna) e la \underline{tipologia} (mountain bike, city bike o elettrica). Per ciascun mezzo, il sistema deve monitorare costantemente lo \underline{stato di disponibilità} (libera, noleggiata o in manutenzione). Ogni \underline{bicicletta} è inoltre equipaggiata con un dispositivo \underline{GPS} (identificato dal suo \underline{numero di serie}) e dispone di una \underline{nota danni preesistenti} per evitare contestazioni alla riconsegna.

\medskip

Le \underline{biciclette elettriche} richiedono una gestione specifica: per esse è necessario registrare lo \underline{stato di carica della batteria} sia al momento della \underline{consegna}, sia al momento della \underline{restituzione}, per far eventualmente pagare la differenza di carica al cliente.

\medskip

Il negozio interagisce con i \underline{clienti}, i quali sono identificati dal proprio \underline{codice fiscale} e dei quali si conservano le generalità come \underline{nome}, \underline{cognome} e un \underline{recapito telefonico}. Il \underline{cliente} può sottoscrivere uno o più \underline{noleggi} nel tempo. Ogni \underline{noleggio} associa formalmente una \underline{bicicletta} a un \underline{cliente} per un determinato \underline{periodo di utilizzo}.

\medskip

In fase di stipula del contratto, viene applicata una \underline{tariffa} che dipende dalla \underline{durata del noleggio} e dal \underline{tipo di bici}. La durata del noleggio segue delle \underline{formule standard}: giornaliera, settimanale o mensile. Per ogni operazione conclusa, il sistema deve tenere traccia dei dettagli del \underline{pagamento}, memorizzando il \underline{codice ricevuta} emesso e la \underline{modalità di pagamento} (contanti, carta o bitcoin).

\subsection{Glossario dei Termini}

\begin{table}[h]
\centering
\begin{tabular}{|l|p{6cm}|l|}
\hline
\textbf{Nome} & \textbf{Descrizione} & \textbf{Collegamenti} \\ \hline
Bici & unità fisica della flotta, identificata dal codice telaio e dotata di GPS. & Cliente, Noleggio \\ \hline
Bici Elettrica & Specializzazione della bicicletta che richiede il monitoraggio della batteria. & Bici, Noleggio \\ \hline
Cliente & Soggetto che richiede il servizio, identificato dal codice fiscale. & Noleggio \\ \hline
Noleggio & Operazione che lega un cliente a una bicicletta per un determinato periodo di tempo. & Cliente, Bici, Pagamento, Tariffa \\ \hline
Tariffa & Costo dipendente da durata e tipologia di bicicletta. & Noleggio \\ \hline
Pagamento & Transazione economica associata a un noleggio, identificata dal codice ricevuta. & Noleggio \\ \hline
\end{tabular}
\end{table}

\subsection{Costruzione Schema Concettuale}

Di seguito si mostra un primo schema E-R:

\begin{figure}[H]
    \centering
    \includegraphics[width=1\textwidth]{assets/schema1.pdf}
\end{figure}

\section{Progettazione Logica}
\subsection{Analisi delle Prestazioni dello Schema Concettuale}

\subsubsection{Tabella dei volumi}

Si riporta di seguito una tabella dei volumi indicativa, prendendo come riferimento 1000 noleggi.

\begin{table}[h]
\centering
\renewcommand{\arraystretch}{1.5}
\begin{tabular}{|l|c|r|}
\hline
\textbf{Concetto} & \textbf{Tipo} & \textbf{Volume} \\ \hline
Cliente         & E & 600 \\ \hline
Noleggio        & E & 1.000 \\ \hline
Pagamento       & E & 1.000 \\ \hline
Bici            & E & 100 \\ \hline
Tariffa         & E & 9 \\ \hline
sottoscrive     & R & 1.000 \\ \hline
utilizza        & R & 1.000 \\ \hline
applica         & R & 1.000 \\ \hline
genera          & R & 1.000 \\ \hline
\end{tabular}
\end{table}

\subsubsection{Insieme delle operazioni}

Si elencano di seguito le principali operazioni che il sistema dovrà supportare:

\begin{enumerate}
    \item registrazione di un \textbf{nuovo cliente}
    \item aggiunta di una \textbf{nuova bicicletta} alla flotta
    \item avvio di un \textbf{nuovo noleggio}
    \item conclusione di un noleggio e registrazione del \textbf{pagamento}
\end{enumerate}

\pagebreak
\subsubsection{Tabelle degli accessi}

Per ogni operazione, si stima il numero di accessi alle entità e relazioni coinvolte.

\begin{table}[H]
\centering
\renewcommand{\arraystretch}{1.5}
\begin{tabular}{|l|c|c|c|c|}
\hline
\textbf{Componente} & \textbf{Reg. Cliente} & \textbf{Agg. Bici} & \textbf{Avvio Nol.[1]} & \textbf{Conc. Nol.[2]} \\ \hline
Cliente                     & 0 & 0 & 0 & 0 \\ \hline
Bici                        & 0 & 0 & 1 & 0 \\ \hline
Noleggio                    & 0 & 0 & 0 & 1 \\ \hline
Pagamento                   & 0 & 0 & 0 & 0 \\ \hline
Tariffa                     & 0 & 0 & 1 & 0 \\ \hline
sottoscrive                 & 0 & 0 & 0 & 0 \\ \hline
utilizza                    & 0 & 0 & 0 & 0 \\ \hline
applica                     & 0 & 0 & 0 & 0 \\ \hline
genera                      & 0 & 0 & 0 & 0 \\ \hline
\rowcolor{yellow!20} Tot L  & 0 & 0 & 2 & 1 \\ \hline
\end{tabular}
\caption{Tabella degli Accessi in Lettura}
\end{table}

\begin{table}[H]
\centering
\renewcommand{\arraystretch}{1.5}
\begin{tabular}{|l|c|c|c|c|}
\hline
\textbf{Componente} & \textbf{Reg. Cliente} & \textbf{Agg. Bici} & \textbf{Avvio Nol.[1]} & \textbf{Conc. Nol.[2]} \\ \hline
Cliente                     & 1 & 0 & 0 & 0 \\ \hline
Bici                        & 0 & 1 & 0 & 0 \\ \hline
Noleggio                    & 0 & 0 & 1 & 1 \\ \hline
Pagamento                   & 0 & 0 & 0 & 1 \\ \hline
Tariffa                     & 0 & 0 & 0 & 0 \\ \hline
sottoscrive                 & 0 & 0 & 1 & 0 \\ \hline
utilizza                    & 0 & 0 & 1 & 0 \\ \hline
applica                     & 0 & 0 & 1 & 0 \\ \hline
genera                      & 0 & 0 & 0 & 1 \\ \hline
\rowcolor{yellow!20} Tot S  & 1 & 1 & 4 & 3 \\ \hline
\end{tabular}
\caption{Tabella degli Accessi in Scrittura}
\end{table}

\smallskip
\noindent\textit{[1] Nota: si ipotizza il noleggio di una bicicletta non elettrica}

\noindent\textit{[2] Nota: si ipotizza che la nota di stato della bicicletta non venga cambiato}


\pagebreak
\subsection{Ristrutturazione Modello E-R}

\subsubsection{Eliminazione delle ridondanze}

\bigskip

\noindent\textbf{Duplicazione degli attributi di costo}

\medskip
Si osserva la presenza di due campi, \texttt{costo} e \texttt{importo}, sia nell'entità \textbf{Tariffa} che nell'entità \textbf{Pagamento}.
Sebbene a primo impatto possa sembrare una \emph{ridondanza}, questa scelta risponde ad un'esigenza progettuale ben precisa.

\medskip
infatti la tabella \textbf{Tariffa} rappresenta un \emph{listino prezzi} soggetto a possibili variazioni nel tempo, mentre l'attributo \texttt{importo} in \textbf{Pagamento} indica il prezzo \emph{effettivamente concordato} per uno specifico noleggio.

\medskip
Questa distinzione consente di mantenere l'\textbf{accuratezza temporale}: anche se il listino viene aggiornato, i pagamenti storici conservano l'importo originariamente applicato.

\bigskip
\hrule
\bigskip

\noindent\textbf{Gestione delle biciclette elettriche}\label{sec:ridondanza_bici_elettriche}

\medskip
Nel primo schema emerge una criticità legata alla gestione delle bici elettriche. Come richiesto dalla specifica, per questa tipologia di biciclette è necessario tracciare lo \textbf{stato di carica della batteria} sia all'inizio che alla fine di ogni noleggio.

\medskip
La soluzione attualmente adottata comporta la presenza di \emph{valori nulli} per tutte le biciclette non elettriche, introducendo una ridondanza. Sfortunatamente, la rimozione di questa ridondanza non è semplice, poiché richiederebbe la creazione di una \textbf{specializzazione} dell'entità \textbf{Bici} ed una relazione aggiuntiva con \textbf{Noleggio} come mostrato di seguito:

\begin{figure}[H]
    \centering
    \includegraphics[width=0.6\textwidth]{assets/schema2.pdf}
\end{figure}
\medskip

Tuttavia, in fase di traduzione verso il modello relazionale, la specializzazione verrebbe comunque rimossa con la strategia di accorpamento nell'entità padre, riportando la situazione alla condizione iniziale.
\medskip
Scelgo perciò di accettare la ridondanza attuale, avendo \emph{valori nulli} per ogni bicicletta non elettrica.

\bigskip
\hrule
\bigskip

\noindent\textbf{Scalabilità del tariffario}

\medskip
Un'ulteriore fonte di ridondanza riguarda il \textbf{tariffario}, che richiede una tupla per ogni combinazione di \texttt{durata} e \texttt{tipologia\_bici}, con complessità spaziale $O(m \times n)$. Sebbene la tabella \textbf{Tariffa} cresca raramente, l'approccio risulta poco scalabile.

\bigskip
\hrule
\bigskip

Di seguito si mostra lo schema E-R finale:

\begin{figure}[H]
    \centering
    \includegraphics[width=1\textwidth]{assets/schema3.pdf}
\end{figure}

\subsubsection{Tabella dei volumi aggiornata}

Si riporta di seguito la tabella dei volumi finale, prendendo come riferimento 1000 noleggi.

\begin{table}[h!]
\centering
\begin{tabular}{|l|c|c|}
\hline
\textbf{Concetto} & \textbf{Tipo} & \textbf{Volume} \\ \hline
Cliente           & E & 600 \\ \hline
Bici              & E & 100 \\ \hline
Tipologia Bici    & E & 3 \\ \hline
Noleggio          & E & 1000 \\ \hline
Pagamento         & E & 1000 \\ \hline
Tariffa Base      & E & 3 \\ \hline
sottoscrive       & R & 1000 \\ \hline
utilizza          & R & 1000 \\ \hline
applica           & R & 1000 \\ \hline
genera            & R & 1000 \\ \hline
appartiene        & R & 100 \\ \hline
\end{tabular}
\end{table}

\subsubsection{Tabelle degli accessi aggiornate}

\begin{table}[H]
\centering
\renewcommand{\arraystretch}{1.5}
\begin{tabular}{|l|c|c|c|c|}
\hline
\textbf{Componente} & \textbf{Reg. Cliente} & \textbf{Agg. Bici} & \textbf{Avvio Nol.[1]} & \textbf{Conc. Nol.[2]} \\ \hline
Cliente                     & 0 & 0 & 0 & 0 \\ \hline
Bici                        & 0 & 0 & 0 & 0 \\ \hline
Tipologia Bici              & 0 & 0 & 1 & 0 \\ \hline
Noleggio                    & 0 & 0 & 0 & 1 \\ \hline
Pagamento                   & 0 & 0 & 0 & 0 \\ \hline
Tariffa Base                & 0 & 0 & 1 & 0 \\ \hline
sottoscrive                 & 0 & 0 & 0 & 0 \\ \hline
utilizza                    & 0 & 0 & 0 & 0 \\ \hline
applica                     & 0 & 0 & 0 & 0 \\ \hline
genera                      & 0 & 0 & 0 & 0 \\ \hline
appartiene                  & 0 & 0 & 1 & 0 \\ \hline
\rowcolor{yellow!20} Tot L  & 0 & 0 & 3 & 1 \\ \hline
\end{tabular}
\caption{Tabella degli Accessi in Lettura}
\end{table}

\begin{table}[H]
\centering
\renewcommand{\arraystretch}{1.5}
\begin{tabular}{|l|c|c|c|c|}
\hline
\textbf{Componente} & \textbf{Reg. Cliente} & \textbf{Agg. Bici} & \textbf{Avvio Nol.[1]} & \textbf{Conc. Nol.[2]} \\ \hline
Cliente                     & 1 & 0 & 0 & 0 \\ \hline
Bici                        & 0 & 1 & 0 & 0 \\ \hline
Noleggio                    & 0 & 0 & 1 & 1 \\ \hline
Pagamento                   & 0 & 0 & 0 & 1 \\ \hline
Tipologia Bici              & 0 & 0 & 0 & 0 \\ \hline
Tariffa Base                & 0 & 0 & 0 & 0 \\ \hline
sottoscrive                 & 0 & 0 & 1 & 0 \\ \hline
genera                      & 0 & 0 & 0 & 1 \\ \hline
utilizza                    & 0 & 0 & 1 & 0 \\ \hline
applica                     & 0 & 0 & 1 & 0 \\ \hline
appartiene                  & 0 & 1 & 0 & 0 \\ \hline
consuma                     & 0 & 0 & 0 & 0 \\ \hline
\rowcolor{yellow!20} Tot S  & 1 & 2 & 4 & 3 \\ \hline
\end{tabular}
\caption{Tabella degli Accessi in Scrittura}
\end{table}

\smallskip
\noindent\textit{[1] Nota: si ipotizza il noleggio di una bicicletta non elettrica}

\noindent\textit{[2] Nota: si ipotizza che la nota di stato della bicicletta non venga cambiato}

\medskip
Le tabelle aggiornate non evidenziano variazioni significative nel numero di accessi, presentando anzi un lieve peggioramento. Tuttavia, la ristrutturazione apporta benefici in termini di \emph{efficienza spaziale}, portando ad una base di dati più scalabile e mantenibile.

\subsubsection{Eliminazione delle generalizzazioni}

Nello schema E-R finale non sono presenti generalizzazioni da eliminare (vedi sez. \ref{sec:ridondanza_bici_elettriche}).

\subsubsection{Rimozione degli attributi composti}

Nello schema E-R finale non sono presenti attributi composti da rimuovere.

\subsubsection{Scelta degli identificatori principali}

Nello schema E-R finale, gli identificatori principali sono stati scelti come segue:
\begin{itemize}
    \item \textbf{Cliente}: codice fiscale
    \item \textbf{Bici}: codice telaio (eventualmente anche numero di serie GPS)
    \item \textbf{Tipologia Bici}: nome
    \item \textbf{Noleggio}: identificatore univoco generato automaticamente (UUID o seriale)
    \item \textbf{Pagamento}: codice ricevuta
    \item \textbf{Tariffa Base}: numero di giorni
\end{itemize}

Per l'entità Noleggio si è scelto di usare un id generato automaticamente perché una chiave composta naturale basata su 
\texttt{data}, \texttt{codice\_fiscale} e \texttt{codice\_telaio} sarebbe poco pratica.

\subsection{Traduzione verso il Modello Relazionale}

Di seguito viene mostrato lo schema relazionale risultante:

\begin{itemize}
    \item \textbf{Noleggio}(\underline{id}, data\_inizio, data\_fine, cliente, tariffa)
    \begin{itemize}
        \item Noleggio.cliente $\rightarrow$ Cliente.codice\_fiscale
        \item Noleggio.tariffa $\rightarrow$ Tariffa\_Base.numero\_giorni
    \end{itemize}

    \item \textbf{Pagamento}(\underline{codice\_ricevuta}, modalita, importo, noleggio)
    \begin{itemize}
        \item Pagamento.noleggio $\rightarrow$ Noleggio.id
    \end{itemize}

    \item \textbf{Bici}(\underline{codice\_telaio}, numero\_serie\_gps, taglia, genere, nota\_danni, stato, tipologia)
    \begin{itemize}
        \item Bici.tipologia $\rightarrow$ Tipologia\_Bici.nome
    \end{itemize}

    \item \textbf{Tariffa\_Base}(\underline{numero\_giorni}, prezzo)
    \item \textbf{Cliente}(\underline{codice\_fiscale}, nome, cognome, cellulare)
    \item \textbf{Tipologia\_Bici}(\underline{nome}, moltiplicatore\_costo)

    \item \textbf{Utilizza}(\underline{noleggio}, \underline{bici}, batteria\_inizio, batteria\_fine)
    \begin{itemize}
        \item Utilizza.noleggio $\rightarrow$ Noleggio.id
        \item Utilizza.bici $\rightarrow$ Bici.codice\_telaio
    \end{itemize}

\end{itemize}

\section{Progettazione Fisica}

\subsection{Creazione Tabelle}

Le tabelle create sono le seguenti: \textbf{Noleggio}, \textbf{Pagamento}, \textbf{Bici}, \textbf{Tariffa\_Base}, \textbf{Cliente}, \textbf{Tipologia\_Bici} e \textbf{Utilizza}.

\medskip
Ogni attributo è stato definito con il tipo di dato più appropriato tenendo in considerazione i casi d'uso e i vincoli di integrità.

\medskip
Inoltre sono stati aggiunti una serie di controlli per sanificare attributi come numero di telefono, prezzi, date, stati, ecc. così da prevenire errori umani ove possibile.

\subsection{Popolamento Tabelle}

Le tabelle sono state popolate con dati arbitrari il più realistici possibile, in modo da poter testare le funzionalità del database.

\subsection{Triggers}

Sono stati ideati 4 triggers per una maggiore integrità dei dati:

\begin{itemize}
    \item \textbf{trg\_chk\_bici\_disponibile}: prima dell'inserimento di una tupla in \textbf{Utilizza}, verifica che la bici sia effettivamente disponibile.
    \item \textbf{trg\_set\_bici\_in\_uso}: dopo l'inserimento di una tupla in \textbf{Utilizza}, aggiorna lo stato della bici a "in uso".
    \item \textbf{trg\_set\_bici\_disponibile}: dopo l'inserimento di una tupla in \textbf{Pagamento}, aggiorna lo stato della bici a "disponibile".
    \item \textbf{trg\_chk\_importo\_pagamento}: prima dell'inserimento di una tupla in \textbf{Pagamento}, verifica che l'importo rispecchi la tariffa applicata.
\end{itemize}

\subsection{Viste}

Sono state ideate 3 viste per facilitare l'accesso a informazioni rilevanti:

\begin{itemize}
    \item \textbf{v\_noleggi\_attivi}: mostra tutti i noleggi attualmente in corso con data di fine prevista e cliente associato, mostrando i più urgenti per primi.
    \item \textbf{v\_clienti\_frequenti}: elenca i clienti abituali, ordinati per numero di noleggi effettuati.
    \item \textbf{v\_bici\_da\_caricare}: elenca le biciclette elettriche che necessitano di ricarica, mostrando prima le più scariche.
\end{itemize}

\subsection{Procedure e Funzioni}

Sono state implementate procedure per automatizzare operazioni comuni:

\begin{itemize}
    \item \textbf{sp\_inizia\_noleggio}: avvia un nuovo noleggio per un cliente e una bici specifica, registrando in automatico la data di inizio e lo stato della bici.
    \item \textbf{sp\_concludi\_noleggio}: conclude un noleggio, aggiornando la data di fine, lo stato della bici e registrando il pagamento.
    \item \textbf{sp\_report\_incassi}: genera un report degli incassi totali prendendo in input un intervallo di date.
\end{itemize}

\end{document}
