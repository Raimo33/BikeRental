\documentclass[12pt,a4paper]{article}

\usepackage[utf8]{inputenc}
\usepackage[main=italian,provide=*]{babel}
\usepackage{graphicx}
\usepackage{amsmath}
\usepackage{hyperref}
\usepackage{geometry}
\usepackage{float}
\geometry{margin=2.5cm}

\setlength{\parindent}{0pt}

\title{Progettazione Database - Noleggio Bici}
\author{Claudio Raimondi}
\date{\today}

\begin{document}

\maketitle

\tableofcontents
\newpage

\section{Richiesta}

Un negozio di noleggio biciclette richiede una base di dati per la gestione operativa della propria flotta, dei clienti e delle transazioni di noleggio effettuate.

\vspace{1em}

Ogni bicicletta è identificata univocamente da un codice telaio ed è caratterizzata da specifiche proprietà tecniche: la taglia(bambino, media o grande), il genere(uomo o donna) e la tipologia(mountain bike, city bike o elettrica). Per ciascun mezzo, il sistema deve monitorare costantemente lo stato di disponibilità(libera, noleggiata o in manutenzione). Ogni bicicletta è inoltre equipaggiata con un dispositivo GPS(identificato dal suo numero di serie) e dispone di una nota danni preesistenti per evitare contestazioni alla riconsegna.

\vspace{1em}

Le biciclette elettriche richiedono una gestione specifica: per esse è necessario registrare lo stato di carica della batteria sia al momento della consegna, sia al momento della restituzione, per far eventualmente pagare la differenza di carica al cliente.

\vspace{1em}

Il negozio interagisce con i clienti, i quali sono identificati dal proprio codice fiscale e dei quali si conservano le generalità come nome, cognome e un recapito telefonico. Il cliente può sottoscrivere uno o più noleggi nel tempo. Ogni noleggio associa formalmente una bicicletta a un cliente per un determinato periodo di utilizzo.

\vspace{1em}

In fase di stipula del contratto, viene applicata una tariffa che dipende dalla durata del noleggio e dal tipo di bici. La durata del noleggio segue delle formule standard: giornaliera, settimanale o mensile. Per ogni operazione conclusa, il sistema deve tenere traccia dei dettagli del pagamento, memorizzando il codice ricevuta emesso e la modalità di pagamento(contanti, carta o bitcoin).

\section{Progettazione Concettuale}

\subsection{Analisi Richiesta}

Di seguito si analizza la richiesta evidenziando i termini chiave.

\vspace{1em}

Ogni \underline{bicicletta} è identificata univocamente da un \underline{codice telaio} ed è caratterizzata da specifiche proprietà tecniche: la \underline{taglia} (bambino, media o grande), il \underline{genere} (uomo o donna) e la \underline{tipologia} (mountain bike, city bike o elettrica). Per ciascun mezzo, il sistema deve monitorare costantemente lo \underline{stato di disponibilità} (libera, noleggiata o in manutenzione). Ogni \underline{bicicletta} è inoltre equipaggiata con un dispositivo \underline{GPS} (identificato dal suo \underline{numero di serie}) e dispone di una \underline{nota danni preesistenti} per evitare contestazioni alla riconsegna.

\vspace{1em}

Le \underline{biciclette elettriche} richiedono una gestione specifica: per esse è necessario registrare lo \underline{stato di carica della batteria} sia al momento della \underline{consegna}, sia al momento della \underline{restituzione}, per far eventualmente pagare la differenza di carica al cliente.

\vspace{1em}

Il negozio interagisce con i \underline{clienti}, i quali sono identificati dal proprio \underline{codice fiscale} e dei quali si conservano le generalità come \underline{nome}, \underline{cognome} e un \underline{recapito telefonico}. Il \underline{cliente} può sottoscrivere uno o più \underline{noleggi} nel tempo. Ogni \underline{noleggio} associa formalmente una \underline{bicicletta} a un \underline{cliente} per un determinato \underline{periodo di utilizzo}.

\vspace{1em}

In fase di stipula del contratto, viene applicata una \underline{tariffa} che dipende dalla \underline{durata del noleggio} e dal \underline{tipo di bici}. La durata del noleggio segue delle \underline{formule standard}: giornaliera, settimanale o mensile. Per ogni operazione conclusa, il sistema deve tenere traccia dei dettagli del \underline{pagamento}, memorizzando il \underline{codice ricevuta} emesso e la \underline{modalità di pagamento} (contanti, carta o bitcoin).

\subsection{Glossario dei Termini}

\begin{table}[h]
\centering
\begin{tabular}{|l|p{6cm}|l|}
\hline
\textbf{Nome} & \textbf{Descrizione} & \textbf{Collegamenti} \\ \hline
Bicicletta & unità fisica della flotta, identificata dal codice telaio e dotata di GPS. & Cliente, Noleggio \\ \hline
Bicicletta Elettrica & Specializzazione della bicicletta che richiede il monitoraggio della batteria. & Bicicletta, Noleggio \\ \hline
Cliente & Soggetto che richiede il servizio, identificato dal codice fiscale. & Noleggio \\ \hline
Noleggio & Operazione che lega un cliente a una bicicletta per un determinato periodo di tempo. & Cliente, Bicicletta, Pagamento, Tariffa \\ \hline
Tariffa & Costo dipendente da durata e tipologia di bicicletta. & Noleggio \\ \hline
Pagamento & Transazione economica associata a un noleggio, identificata dal codice ricevuta. & Noleggio \\ \hline
\end{tabular}
\end{table}

\subsection{Costruzione Schema Concettuale}

Di seguito si mostra un primo schema E-R:

\begin{figure}[H]
    \centering
    \includegraphics[width=1\textwidth]{assets/schema1.pdf}
    \label{fig:schema1}
\end{figure}

\section{Progettazione Logica}
\subsection{Analisi delle Prestazioni dello Schema Concettuale}


\subsubsection{Tabella dei volumi}
\subsubsection{Insieme delle operazioni}
\subsubsection{Tabelle degli accessi}
\subsection{Ristrutturazione Modello E-R}

\subsubsection{Eliminazione delle ridondanze}

\bigskip

\noindent\textbf{Duplicazione degli attributi di costo}

\medskip
Si osserva la presenza di due campi, \texttt{costo} e \texttt{importo}, sia nell'entità \textbf{Tariffa} che nell'entità \textbf{Pagamento}.  
Sebbene a primo impatto possa sembrare una \emph{ridondanza}, questa scelta risponde ad un'esigenza progettuale ben precisa.

\medskip
\begin{itemize}
    \item la tabella \textbf{Tariffa} rappresenta un \emph{listino prezzi} soggetto a possibili variazioni nel tempo
    \item l'attributo \texttt{importo} in \textbf{Pagamento} indica il prezzo \emph{effettivamente concordato} per uno specifico noleggio
\end{itemize}

\medskip
Questa distinzione consente di mantenere l'\textbf{accuratezza temporale}: anche se il listino viene aggiornato, i pagamenti storici conservano l'importo originariamente applicato.

\bigskip
\hrule
\bigskip

\noindent\textbf{Gestione delle biciclette elettriche}

\medskip
Nel primo schema emerge una criticità legata alla gestione delle bici elettriche. Come richiesto dalla specifica, per questa tipologia di biciclette è necessario tracciare lo \textbf{stato di carica della batteria} sia all'inizio che alla fine di ogni noleggio.

\medskip
La soluzione attualmente adottata comporta la presenza di \emph{valori nulli} per tutte le biciclette non elettriche, introducendo una ridondanza non necessaria.

\bigskip
\hrule
\bigskip

\noindent\textbf{Scalabilità del tariffario}

\medskip
Un'ulteriore fonte di ridondanza riguarda il \textbf{tariffario}, che richiede una tupla per ogni combinazione di \texttt{durata} e \texttt{tipologia\_bici}, con complessità spaziale $O(m \times n)$. Sebbene la tabella \textbf{Tariffa} creasca raramente, l'approccio risulta poco scalabile.

\bigskip
\hrule
\bigskip

Di seguito si mostra lo schema E-R finale:

\begin{figure}[H]
    \centering
    \includegraphics[width=1\textwidth]{assets/schema2.pdf}
    \label{fig:schema2}
\end{figure}

\subsubsection{Eliminazione delle generalizzazioni}
\subsubsection{Rimozione degli attributi composti}
\subsubsection{Scelta degli identificatori principali}
\subsubsection{Risultato finale della ristrutturazione}
\subsection{Traduzione verso il Modello Relazionale}

\section{Progettazione Fisica}
\subsection{Creazione e Popolamento Tabelle}
\subsection{Triggers}
\subsection{Interrogazioni}
\subsection{Procedure e Funzioni}

% aggiornamento tariffa con base * moltiplicatore

\subsection{Viste}

\end{document}
